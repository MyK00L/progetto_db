\documentclass[a4paper,12pt]{report}

\usepackage[italian]{babel}
\usepackage[utf8]{inputenc}
\usepackage[T1]{fontenc}
%\usepackage[style=numeric-comp]{biblatex}

\title{InfoTreno}
\author{Chelli M. \thanks{michael.chelli@studio.unibo.it - 915585}, Tampieri E.\thanks{eugenio.tampieri@studio.unibo.it - 915602} - Gruppo 2098}

\begin{document}
	\maketitle
	\tableofcontents
	\chapter{Analisi dei requisiti}
	\section{Intervista}
	\par RFS (Rete Ferroviaria dello Stato) richiede la realizzazione di un sistema informativo in grado di monitorare la marcia e la programmazione dei treni e la gestione dei turni del personale di bordo. Viene richiesta la possibilità di operare tramite interfaccia web, in modo da essere indipendenti dalle piattaforme utilizzate.
	\par
	\section{Rilevamento delle ambiguità e correzioni proposte}
	\subsection{Dopo la correzione delle ambiguità}
	\section{Definizione delle specifiche in linguaggio naturale ed estrazione dei concetti principali}
	\chapter{Progettazione concettuale}
	\section{Schema scheletro}
	\section{Raffinamenti proposti}
	\subsection{Entità ...}
	\section{Schema concettuale finale}
	\chapter{Progettazione logica}
	\section{Stima del volume dei dati}
	\section{Descrizione delle operazioni principali e stima della loro frequenza}
	\section{Schemi di navigazione e tabelle degli accessi}
	\subsection{Schemi di navigazione}
	\subsection{Tabella degli accessi}
	\section{Raffinamento dello schema (eliminazione di identificatori esterni, attributi composti e gerarchie, scelta delle chiavi)}
	\subsection{Eliminazione delle gerarchie}
	\subsection{Attributi composti}
	\subsection{Scelta delle chiavi primarie}
	\subsection{Chiavi esterne}
	\subsection{Vincoli di gruppo}
	\subsection{Accorgimenti}
	\section{Analisi delle ridondanze}
	\section{Traduzione di entità e associazioni in relazioni}
	\section{Schema relazionale finale}
	\section{Traduzione delle operazioni in query SQL}
	\chapter{Progettazione dell'applicazione}
	\section{Descrizione dell'architettura dell'applicazione realizzata}
	\subsection{Homepage}
	\subsection{Ricerca treno}
	\subsection{Ricerca stazione...}
    %\printbibliography[heading=bibintoc]
\end{document}
