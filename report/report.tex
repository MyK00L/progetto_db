\documentclass[a4paper,12pt]{report}

\usepackage[italian]{babel}
\usepackage[utf8]{inputenc}
\usepackage[T1]{fontenc}
%\usepackage[style=numeric-comp]{biblatex}

\title{InfoTreno}
\author{Chelli M. \thanks{michael.chelli@studio.unibo.it - 915585}, Tampieri E.\thanks{eugenio.tampieri@studio.unibo.it - 915602} - Gruppo 2098}

\begin{document}
	\maketitle
	\tableofcontents
	\chapter{Analisi dei requisiti}
	\section{Intervista}
	\par RFS (Rete Ferroviaria dello Stato) richiede la realizzazione di un sistema informativo in grado di monitorare la marcia e la programmazione dei treni e la gestione dei turni del personale di bordo. Viene richiesta la possibilità di operare tramite interfaccia web, in modo da essere indipendenti dalle piattaforme utilizzate.
	\par Un treno è uno specifico viaggio su una relazione, ovvero l'attraversamento sequenziale di una serie di punti di passaggio (scambi, stazioni, o semplici) in orari predeterminati.
	\par Oltre alla memorizzazione degli orari di attraversamento teorico, viene richiesta la memorizzazione della data e ora di partenza e di arrivo da un punto di passaggio, così da poter calcolare il ritardo del treno.
	\par Un treno è poi composto da una locomotiva (della quale ci interessa conoscere la velocità e la tensione di esercizio) e una serie di carrozze (delle quali ci interessa memorizzare la classe e il numero di posti), che formano un convoglio.
	\par Su un treno prendono servizio un macchinista, un capotreno e, in certi casi, dei controllori.
	\section{Rilevamento delle ambiguità e correzioni proposte}
	\subsection{Dopo la correzione delle ambiguità}
	\section{Definizione delle specifiche in linguaggio naturale ed estrazione dei concetti principali}
	\par Si individuano le parole chiave che permetteranno di costruire un primo schema scheletro del progetto. In seguito sarà poi raffinato per ottenere lo schema definitivo. I termini essenziali sono evidenziati in grassetto e in corsivo.
	\chapter{Progettazione concettuale}
	\par In questa situazione il dominio è specifico ed è spesso possibile trovare delle entità generiche che si concretizzano poi in entità più specifiche. Per questo motivo è utile utilizzare un approccio top-down per modellare al meglio l'architettura.
	\section{Schema scheletro}
	\section{Raffinamenti proposti}
	\subsection{Entità ...}
	\section{Schema concettuale finale}
	\chapter{Progettazione logica}
	\section{Stima del volume dei dati}
	\par Si ripoerao le stime dei volumi dei dati dopo un anno di operatività.
	\begin{table}
	\centering
	\begin{tabular}{|l|l|l|}
		\hline
		Nome & Tipo & Volume \\
		\hline
		     &      &        \\
		\hline
		     &      &        \\
		\hline
		     &      &        \\
		\hline
	\end{tabular}
	\end{table}
	\section{Descrizione delle operazioni principali e stima della loro frequenza}
	\section{Schemi di navigazione e tabelle degli accessi}
	\subsection{Schemi di navigazione}
	\subsection{Tabella degli accessi}
	\section{Raffinamento dello schema (eliminazione di identificatori esterni, attributi composti e gerarchie, scelta delle chiavi)}
	\subsection{Eliminazione delle gerarchie}
	\subsection{Attributi composti}
	\subsection{Scelta delle chiavi primarie}
	\subsection{Chiavi esterne}
	\subsection{Vincoli di gruppo}
	\subsection{Accorgimenti}
	\section{Analisi delle ridondanze}
	\section{Traduzione di entità e associazioni in relazioni}
	\section{Schema relazionale finale}
	\section{Traduzione delle operazioni in query SQL}
	\chapter{Progettazione dell'applicazione}
	\section{Descrizione dell'architettura dell'applicazione realizzata}
	\subsection{Homepage}
	\subsection{Ricerca treno}
	\subsection{Ricerca stazione...}
	\appendix
	\chapter{Istruzioni per l'installazione}
	\section{Prerequisiti}
	\begin{itemize}
		\item Un DBMS PostgreSQL con già creato un database vuoto
		\item Un utente che abbia i permessi di creazione tabelle e viste, di interrogazione e di inserimento sul database
	\end{itemize}
	\section{Configurazione}
	\par \`E necessario creare un file denominato \texttt{.env} nella cartella in cui verrà eseguito il programma, contenente nel formato \texttt{<KEY>=<VALUE>} (una variabile per riga) le seguenti variabili:
	\begin{itemize}
		\item \texttt{DB\_HOST}: l'indirizzo IP o l'hostname su cui è raggiungibile il DBMS,
		\item \texttt{DB\_NAME}: il nome del DB in cui verranno create le tabelle,
		\item \texttt{DB\_USER}: l'utente per autenticarsi al DB,
		\item \texttt{DB\_PASSWORD}: la password dell'utente sopra specificato.
	\end{itemize}
	\section{Esecuzione del programma}
	\par Dopo aver scaricato l'eseguibile per la piattaforma desiderata, controllare che abbia i permessi di esecuzione (se richiesti) ed eseguirlo nella cartella in cui è stato posizionato il file \texttt{.env} e la cartella \texttt{templates}.
	\par Il programma provvederà all'inizializzazione del DB e si metterà in ascolto sulla porta TCP 8000.
    %\printbibliography[heading=bibintoc]
\end{document}
