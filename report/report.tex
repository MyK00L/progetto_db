\documentclass[a4paper,12pt]{report}

\usepackage[italian]{babel}
\usepackage[utf8]{inputenc}
\usepackage[T1]{fontenc}
\usepackage{graphicx}
%\usepackage[style=numeric-comp]{biblatex}

\title{InfoTreno}
\author{Chelli M. \thanks{michael.chelli@studio.unibo.it - 915585}, Tampieri E.\thanks{eugenio.tampieri@studio.unibo.it - 915602} - Gruppo 2098}

\begin{document}
	\maketitle
	\tableofcontents
	\chapter{Analisi dei requisiti}
	\section{Intervista}
	\par RFS (Rete Ferroviaria dello Stato) richiede la realizzazione di un sistema informativo in grado di monitorare la marcia e la programmazione dei treni e la gestione dei turni del personale di bordo. Viene richiesta la possibilità di operare tramite interfaccia web, in modo da essere indipendenti dalle piattaforme utilizzate.
	\par Un treno è uno specifico viaggio su una relazione, ovvero l'attraversamento sequenziale di una serie di punti di passaggio (scambi, stazioni, o semplici) in orari predeterminati.
	\par Oltre alla memorizzazione degli orari di attraversamento teorici, viene richiesta la memorizzazione della data e ora di partenza e di arrivo da un punto di passaggio, così da poter calcolare il ritardo del treno.
	\par Un treno è poi composto da una locomotiva (della quale ci interessa conoscere la velocità e la tensione di esercizio) e una serie di carrozze (delle quali ci interessa memorizzare la classe e il numero di posti), che formano un convoglio.
	\par Su un treno prendono servizio un macchinista, un capotreno e, in certi casi, dei controllori.

	\section{Rilevamento delle ambiguità e correzioni proposte}
	\par RFS (Rete Ferroviaria dello Stato) richiede la realizzazione di un sistema informativo in grado di monitorare la marcia e la programmazione dei treni e la gestione dei turni del personale di bordo. Viene richiesta la possibilità di operare\textsuperscript{1} tramite interfaccia web, in modo da essere indipendenti dalle piattaforme utilizzate.
	\par Un treno è uno specifico viaggio su una relazione\textsuperscript{2}, ovvero l'attraversamento sequenziale di una serie di punti di passaggio\textsuperscript{3} (scambi, stazioni, o semplici\textsuperscript{4}) in orari predeterminati.
	\par Oltre alla memorizzazione degli orari di attraversamento teorico, viene richiesta la memorizzazione della data e ora di partenza e di arrivo da un punto di passaggio, così da poter calcolare il ritardo del treno.
	\par Un treno è poi composto da una locomotiva (della quale ci interessa conoscere la velocità e la tensione di esercizio) e una serie di carrozze (delle quali ci interessa memorizzare la classe\textsuperscript{5} e il numero di posti), che formano un convoglio.
	\par Su un treno prendono servizio un macchinista, un capotreno e, in certi casi, dei controllori.

	\begin{tabular}{|p{1cm}|p{3cm}|p{4cm}|p{4cm}|}
		\hline
		Num & Espressione & Sostituzione & Motivazione \\ \hline
		1 & operare & operare il sistema & specificato il soggetto \\ \hline
		2 & relazione & linea ferroviaria & termine corretto \\ \hline
		3 & punti di passaggio & rappresentati nel mondo fisico da eurobalise & specificato il significato \\ \hline
		4 & semplici & altri & specificato il significato \\ \hline
		5 & classe & classe (prima o seconda) & specificato is significato \\ \hline
	\end{tabular}

	\subsection{Dopo la correzione delle ambiguità}
	\par RFS (Rete Ferroviaria dello Stato) richiede la realizzazione di un sistema informativo in grado di monitorare la marcia e la programmazione dei treni e la gestione dei turni del personale di bordo. Viene richiesta la possibilità di operare is sistema tramite interfaccia web, in modo da essere indipendenti dalle piattaforme utilizzate.
	\par Un treno è uno specifico viaggio su una linea ferroviaria, ovvero l'attraversamento sequenziale di una serie di punti di passaggio, rappresentati nel mondo fisico da eurobalise (scambi, stazioni, o altri) in orari predeterminati.
	\par Oltre alla memorizzazione degli orari di attraversamento teorico, viene richiesta la memorizzazione della data e ora di partenza e di arrivo da un punto di passaggio, così da poter calcolare il ritardo del treno.
	\par Un treno è poi composto da una locomotiva (della quale ci interessa conoscere la velocità e la tensione di esercizio) e una serie di carrozze (delle quali ci interessa memorizzare la classe (prima o seconda) e il numero di posti), che formano un convoglio.
	\par Su un treno prendono servizio un macchinista, un capotreno e, in certi casi, dei controllori.

	\section{Definizione delle specifiche in linguaggio naturale ed estrazione dei concetti principali}
	\par Si individuano le parole chiave che permetteranno di costruire un primo schema scheletro del progetto. In seguito sarà poi raffinato per ottenere lo schema definitivo. I termini essenziali sono evidenziati in grassetto e in corsivo.
	\par RFS (Rete Ferroviaria dello Stato) richiede la realizzazione di un sistema informativo in grado di monitorare la marcia e la programmazione dei treni e la gestione dei turni del personale di bordo. Viene richiesta la possibilità di operare is sistema tramite interfaccia web, in modo da essere indipendenti dalle piattaforme utilizzate.
	\par Un \textbf{\textit{treno}} è uno specifico viaggio su una linea ferroviaria, ovvero l'attraversamento sequenziale di una serie di \textbf{\textit{punti di passaggio}}, rappresentati nel mondo fisico da eurobalise (scambi, stazioni, o altri) in orari predeterminati.
	\par Oltre alla memorizzazione degli orari di \textbf{\textit{attraversamento}} teorico, viene richiesta la memorizzazione della data e ora di partenza e di arrivo da un punto di passaggio, così da poter calcolare il \textbf{\textit{ritardo}} del treno.
	\par Un treno è poi composto da una \textbf{\textit{locomotiva}} (della quale ci interessa conoscere la velocità e la tensione di esercizio) e una serie di \textbf{\textit{carrozze}} (delle quali ci interessa memorizzare la classe (prima o seconda) e il numero di posti), che formano un \textbf{\textit{convoglio}}.
	\par Su un treno prendono servizio un macchinista, un capotreno e, in certi casi, dei controllori.


	\chapter{Progettazione concettuale}
	\section{Schema scheletro}
	\section{Raffinamenti proposti}
	\subsection{Entità ...}
	\section{Schema concettuale finale}
	\chapter{Progettazione logica}
	\section{Stima del volume dei dati}
	\par Si ripoerao le stime dei volumi dei dati dopo un anno di operatività.
	\begin{table}
	\centering
	\begin{tabular}{|l|l|l|}
		\hline
		Nome & Tipo & Volume \\
		\hline
		     &      &        \\
		\hline
		     &      &        \\
		\hline
		     &      &        \\
		\hline
	\end{tabular}
	\end{table}
	\section{Descrizione delle operazioni principali e stima della loro frequenza}
	\section{Schemi di navigazione e tabelle degli accessi}
	\subsection{Schemi di navigazione}
	\subsection{Tabella degli accessi}
	\section{Raffinamento dello schema (eliminazione di identificatori esterni, attributi composti e gerarchie, scelta delle chiavi)}
	\subsection{Eliminazione delle gerarchie}
	\subsection{Attributi composti}
	\subsection{Scelta delle chiavi primarie}
	\subsection{Chiavi esterne}
	\subsection{Vincoli di gruppo}
	\subsection{Accorgimenti}
	\section{Analisi delle ridondanze}
	\section{Traduzione di entità e associazioni in relazioni}
	\section{Schema relazionale finale}
	\section{Traduzione delle operazioni in query SQL}
	\chapter{Progettazione dell'applicazione}
	
	\section{Descrizione dell'architettura dell'applicazione realizzata}
	\subsection{Homepage}
	\subsection{Ricerca treno}
	\subsection{Ricerca stazione...}
    %\printbibliography[heading=bibintoc]
\end{document}
